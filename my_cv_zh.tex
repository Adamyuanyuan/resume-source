\documentclass[11pt,a4paper,sans]{moderncv}

% moderncv 主题
\moderncvstyle{classic}
\moderncvcolor{blue}

% 字符编码
\usepackage[utf8]{inputenc}
\usepackage{CJKutf8}

% 调整页面
\usepackage[scale=0.75]{geometry}
%\setlength{\hintscolumnwidth}{3cm}

% 个人信息
\firstname{潘}
\familyname{培}
\title{简历}
\mobile{+86~15057478457}
\email{pannpei@gmail.com}
\homepage{http://xidianpanpei.github.io/}
\nopagenumbers{}  

%----------------------------------------------------------------------------------
%            内容
%----------------------------------------------------------------------------------
\begin{document}
\begin{CJK}{UTF8}{gbsn}
\maketitle

\section{教育背景}
\cventry{2007-2011}{应用物理学学士}{西安电子科技大学,理学院}{}{}{}
\cventry{2013-现在}{软件工程硕士}{浙江大学,软件学院}{}{}{}

\section{毕业论文}
\cvitem{题目}{\emph{单片机和磁阻传感器实现电子指南针}}
\cvitem{导师}{张昌民}
\cvitem{说明}{\small 电子指南针系统采用专用的磁场传感器,结合高速微处理器以及液晶显示器研制而成。系统采用磁阻传感器采集某一方向上的磁场强度之后,通过微处理器对其进行相关的数据处理并在液晶显示系统上显示。}

%\section{实习背景}
%\cventry{year}{Name}{a}{b}{c}{d}

\section{主页}
\cventry{个人主页}{\url{http://xidianpanpei.github.io/}}{}{}{}{}
\cventry{GitHub}{\url{https://github.com/xidianpanpei}}{}{}{}{}
\cventry{新浪微博}{\url{http://weibo.com/u/1951979093}}{}{}{}{}

%\section{出版物}
%\cventry{year}
%{\textbf{Your Name}\textnormal{,Other authors}}
%{Title}{Magazine Name}
%{a}{b}{c}

\section{工作背景}
\subsection{华为技术有限公司}
\cventry{2011.7-2011.10}{云计算设计部}{系统工程师}{}{}{}
\cventry{2011.10-2012.8}{云管理应用开发部}{软件工程师}{}{}{}

\section{项目经历}
\cventry{2011.8-2012.4}{Galax8800 OMM}{Java, BashShell, PostgreSQL}{商业项目}{}{Galax8800 OMM~为大型私有云云管理系统,该系统意在为企业提供大型的私有云解决方案,实现企业数据处理和办公的云端化。\\
在参与该项目过程中,我主要承担的责任为:\\
1.参与告警邮件通知功能设计和开发;\\
2.参与告警功能核心代码编写和交付。}
\vspace*{0.2\baselineskip}
\cventry{2012.4-2012.8}{FusionCube GalaxManager}{Java, BashShell, PostgreSQL, Maven}{商业项目}{}{FusionCube~为华为一体机解决方案项目,此方案意在为中小型企业提供一体机的小型私有云的构建方案,企业可以利用该套解决方案快速构建企业的私有云系统,实现企业办公的云端化。\\
在参与该项目过程中,我主要承担的责任为:\\
1.负责~IAM~部分功能的开发,如登陆校验,规则处理等;\\
2.负责域管理功能的开发以及同其他功能模块的对接;\\
3.负责~IAM~模块的打包安装,使用~Bash、Maven~实现;\\
4.主动承担带领新员工熟悉项目功能并让新员工参与项目开发的任务。}
\vspace*{0.2\baselineskip}
\cventry{2013.9-现在}{Pica Project}{Python, BashShell, OpenStack}{研究项目}{}{Pica~项目是由~Cisco~公司和浙江大学的合作项目,由~Cisco~公司提供资金设备供浙江大学学生进行~OpenStack~相关领域的研究,本项目组主要有网络、存储和调度三个研究方向。本人在项目组中承担网络方向的研究任务,主要研究~OpenStack~中~Neutron~组件的研究,同时,承担了~OpenStack~系统部署任务,现已 成功部署~OpenStack Havana~版本,并向学院内部提供虚拟服务。现阶段主要研究方向为~IPv6~在~OpenStack~中的应用和实现。}

\section{获奖经历}
\cventry{2008}{西安电子科技大学二等奖学金及优秀学生称号}{}{}{}{}

\section{编程技能}
\cventry{编程语言}{Java > BashShell > Python = C = C++ > JavaScript = CSS}{}{}{}{}
\cventry{内核}{Linux}{}{}{}{}
\cventry{数据库}{PostgreSQL = MySQL > SQLite = MongoDB}{}{}{}{}

\section{语言技能}
\cvitemwithcomment{英语}{CET-4 523}{}
\cvitemwithcomment{英语}{TOEIC 665}{}

\section{个人兴趣}
\cvitem{羽毛球}{\small 个人喜好羽毛球,在校期间经常参与学校的相关运动。}
\cvitem{骑行}{\small 喜爱骑行,经常同朋友骑行单车出游。}
\cvitem{编程}{\small 喜欢键盘敲击的感觉,喜欢屏幕上充满优美代码的感觉,喜欢程序调试成功的感觉,喜欢深究之后解决~bug~的感觉。热爱技术,喜欢在网上逛各种技术论坛,了解各种新技术、新工具。}

\clearpage\end{CJK}
\end{document}
